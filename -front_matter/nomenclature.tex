\chapter{Nomenclature}

This nomenclature uses the \mintinline{tex}!longtable! environment
to create a table that can spread over multiple pages.
Alternatively, Overleaf suggests using \texttt{nomencl}.%
\footnote{\url{https://www.overleaf.com/learn/latex/Nomenclatures}}


\section{Latin capitals}

\begin{longtable}[l]{lll}
\textbf{symbol} & \textbf{units} & \textbf{description}\\
\endfirsthead
\textbf{symbol} & \textbf{units} & \textbf{description}\\
\endhead
$G$ & \unit{\metre\cubed\per\kilogram\per\second\squared} & Gravitational constant\\
$T$ & \unit{\kelvin} & Temperature\\
\end{longtable}

\section{Latin lower case}

\begin{longtable}[l]{lll}
\textbf{symbol} & \textbf{units} & \textbf{description}\\
\endfirsthead
\textbf{symbol} & \textbf{units} & \textbf{description}\\
\endhead
$\boldsymbol{g}$ & \unit{\metre\per\second\squared} & Gravitational acceleration\\
$p$ & \unit{\pascal} & Pressure\\
$t$ & \unit{\second} & Time\\
$\boldsymbol{u}$ & \unit{\meter\per\second} & Velocity\\
\end{longtable}

\section{Greek upper case}

\begin{longtable}[l]{lll}
\textbf{symbol} & \textbf{units} & \textbf{description}\\
\endfirsthead
\textbf{symbol} & \textbf{units} & \textbf{description}\\
\endhead
$\varOmega$ & \unit{\radian\per\second} & Angular velocity\\
\end{longtable}

\section{Greek lower case}

\begin{longtable}[l]{lll}
\textbf{symbol} & \textbf{units} & \textbf{description}\\
\endfirsthead
\textbf{symbol} & \textbf{units} & \textbf{description}\\
\endhead
$\mu$ & \unit{\pascal\second} & Dynamic viscosity\\
$\rho$ & \unit{\kilo\gram\per\metre\cubed} & Density\\
\end{longtable}
