\section{Roboto}
\roboto

\dropcap T{he} TU Delft style prescribes Roboto Slab together with Arial.%
\footnote{\url{https://www.tudelft.nl/huisstijl/bouwstenen/typografie}}
You are free to either use Arial (\TeX\ Gyre Heros\footnote{\url{https://ctan.org/pkg/tex-gyre-heros}} when using Pdf\LaTeX) using the class option \texttt{usearial}; else regular Roboto is instead.\footnote{Some significant differences between these otherwise very similar fonts are shown here: \url{http://www.identifont.com/differences?first=Arial&second=Roboto}}
Not surprisingly, this also looks great together with Roboto Slab.
In any case, this section is typeset using Roboto.

Roboto (Sans) has a dual nature. It has a mechanical skeleton
and the forms are largely geometric. At the same time,
the font features friendly and open curves. While some
grotesks distort their letterforms to force a rigid
rhythm, Roboto doesn't compromise, allowing letters to be
settled into their natural width. This makes for a more
natural reading rhythm more commonly found in humanist and
serif types.

Roboto Serif is designed to create a comfortable
and frictionless reading experience. Minimal and
highly functional, it is useful anywhere (even for app
interfaces) due to the extensive set of weights and widths
across a broad range of optical sizes. While it was
carefully crafted to work well in digital media, across
the full scope of sizes and resolutions we have today, it
is just as comfortable to read and work in print media.

Roboto has several styles of digits:
\begin{itemize}
  \itemsep 0pt
  \parskip 0pt
  \item `Normal' lining numbers
  \begin{itemize}
    \ifluatex
      \item Proportional: 1234567890
      \item Tabular: {%  
        \fontspec[Numbers=Monospaced]{Roboto}%
        1234567890%
      }
    \else
      \item Proportional: \robotoLF{1234567890}
      \item Tabular: \robotoTLF{1234567890}
    \fi
  \end{itemize}
  \item Old-style numbers
  \begin{itemize}
    \ifluatex
      \item Proportional: {%
        \fontspec[Numbers=OldStyle]{Roboto}%
        1234567890%
      }
      \item Tabular: {%
        \fontspec[Numbers={OldStyle,Monospaced}]{Roboto}%
        1234567890%
      }
    \else
      \item Proportional: \robotoOsF{1234567890}
      \item Tabular: \robotoTOsF{1234567890}
    \fi
  \end{itemize}
\end{itemize}

Furthermore, the font is available in many different weights:
\begin{itemize}
    \itemsep 0pt
    \parskip 0pt
    \ifluatex
        \item \fontspec[Instance=Thin]{Roboto} Thin
        \item \fontspec[Instance=Extralight]{Roboto} Extralight
        \item \fontspec[Instance=Light]{Roboto} Light
        \item \fontspec[Instance=Regular]{Roboto} Regular
        \item \fontspec[Instance=Medium]{Roboto} Medium
        \item \fontspec[Instance=Semibold]{Roboto} Semibold
        \item \fontspec[Instance=Bold]{Roboto} Bold
        \item \fontspec[Instance=Extrabold]{Roboto} Extrabold
        \item \fontspec[Instance=Black]{Roboto} Black
    \else
        \item \thinseries Roboto Thin
        \item \lightseries Roboto Light
        \item \mdseries Roboto Regular
        \item \mediumseries Roboto Medium
        \item \bfseries Roboto Bold
        \item \blackseries Roboto Black
    \fi
\end{itemize}

\ifluatex
{\robotoflex
Finally, RobotoFlex is a variable font with many `axes' for customisation, allowing for very
{\addfontfeatures{
  Renderer=Harfbuzz,
  Scale=1.2, % inverse of relative x height
  RawAxis={
    width=25,
    weight=400,
    XOPQ=175, % thick stroke
    YOPQ=25, % thin stroke
    YTLC=428, % lower case hight; 514 default
    YTDE=-300, % descender depth
    YTAS=800, % ascender height
  }}
  Fancy
} variation.
}
\fi
